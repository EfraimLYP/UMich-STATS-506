% Options for packages loaded elsewhere
\PassOptionsToPackage{unicode}{hyperref}
\PassOptionsToPackage{hyphens}{url}
\PassOptionsToPackage{dvipsnames,svgnames,x11names}{xcolor}
%
\documentclass[
  letterpaper,
  DIV=11,
  numbers=noendperiod]{scrartcl}

\usepackage{amsmath,amssymb}
\usepackage{lmodern}
\usepackage{iftex}
\ifPDFTeX
  \usepackage[T1]{fontenc}
  \usepackage[utf8]{inputenc}
  \usepackage{textcomp} % provide euro and other symbols
\else % if luatex or xetex
  \usepackage{unicode-math}
  \defaultfontfeatures{Scale=MatchLowercase}
  \defaultfontfeatures[\rmfamily]{Ligatures=TeX,Scale=1}
\fi
% Use upquote if available, for straight quotes in verbatim environments
\IfFileExists{upquote.sty}{\usepackage{upquote}}{}
\IfFileExists{microtype.sty}{% use microtype if available
  \usepackage[]{microtype}
  \UseMicrotypeSet[protrusion]{basicmath} % disable protrusion for tt fonts
}{}
\makeatletter
\@ifundefined{KOMAClassName}{% if non-KOMA class
  \IfFileExists{parskip.sty}{%
    \usepackage{parskip}
  }{% else
    \setlength{\parindent}{0pt}
    \setlength{\parskip}{6pt plus 2pt minus 1pt}}
}{% if KOMA class
  \KOMAoptions{parskip=half}}
\makeatother
\usepackage{xcolor}
\setlength{\emergencystretch}{3em} % prevent overfull lines
\setcounter{secnumdepth}{-\maxdimen} % remove section numbering
% Make \paragraph and \subparagraph free-standing
\ifx\paragraph\undefined\else
  \let\oldparagraph\paragraph
  \renewcommand{\paragraph}[1]{\oldparagraph{#1}\mbox{}}
\fi
\ifx\subparagraph\undefined\else
  \let\oldsubparagraph\subparagraph
  \renewcommand{\subparagraph}[1]{\oldsubparagraph{#1}\mbox{}}
\fi

\usepackage{color}
\usepackage{fancyvrb}
\newcommand{\VerbBar}{|}
\newcommand{\VERB}{\Verb[commandchars=\\\{\}]}
\DefineVerbatimEnvironment{Highlighting}{Verbatim}{commandchars=\\\{\}}
% Add ',fontsize=\small' for more characters per line
\usepackage{framed}
\definecolor{shadecolor}{RGB}{241,243,245}
\newenvironment{Shaded}{\begin{snugshade}}{\end{snugshade}}
\newcommand{\AlertTok}[1]{\textcolor[rgb]{0.68,0.00,0.00}{#1}}
\newcommand{\AnnotationTok}[1]{\textcolor[rgb]{0.37,0.37,0.37}{#1}}
\newcommand{\AttributeTok}[1]{\textcolor[rgb]{0.40,0.45,0.13}{#1}}
\newcommand{\BaseNTok}[1]{\textcolor[rgb]{0.68,0.00,0.00}{#1}}
\newcommand{\BuiltInTok}[1]{\textcolor[rgb]{0.00,0.23,0.31}{#1}}
\newcommand{\CharTok}[1]{\textcolor[rgb]{0.13,0.47,0.30}{#1}}
\newcommand{\CommentTok}[1]{\textcolor[rgb]{0.37,0.37,0.37}{#1}}
\newcommand{\CommentVarTok}[1]{\textcolor[rgb]{0.37,0.37,0.37}{\textit{#1}}}
\newcommand{\ConstantTok}[1]{\textcolor[rgb]{0.56,0.35,0.01}{#1}}
\newcommand{\ControlFlowTok}[1]{\textcolor[rgb]{0.00,0.23,0.31}{#1}}
\newcommand{\DataTypeTok}[1]{\textcolor[rgb]{0.68,0.00,0.00}{#1}}
\newcommand{\DecValTok}[1]{\textcolor[rgb]{0.68,0.00,0.00}{#1}}
\newcommand{\DocumentationTok}[1]{\textcolor[rgb]{0.37,0.37,0.37}{\textit{#1}}}
\newcommand{\ErrorTok}[1]{\textcolor[rgb]{0.68,0.00,0.00}{#1}}
\newcommand{\ExtensionTok}[1]{\textcolor[rgb]{0.00,0.23,0.31}{#1}}
\newcommand{\FloatTok}[1]{\textcolor[rgb]{0.68,0.00,0.00}{#1}}
\newcommand{\FunctionTok}[1]{\textcolor[rgb]{0.28,0.35,0.67}{#1}}
\newcommand{\ImportTok}[1]{\textcolor[rgb]{0.00,0.46,0.62}{#1}}
\newcommand{\InformationTok}[1]{\textcolor[rgb]{0.37,0.37,0.37}{#1}}
\newcommand{\KeywordTok}[1]{\textcolor[rgb]{0.00,0.23,0.31}{#1}}
\newcommand{\NormalTok}[1]{\textcolor[rgb]{0.00,0.23,0.31}{#1}}
\newcommand{\OperatorTok}[1]{\textcolor[rgb]{0.37,0.37,0.37}{#1}}
\newcommand{\OtherTok}[1]{\textcolor[rgb]{0.00,0.23,0.31}{#1}}
\newcommand{\PreprocessorTok}[1]{\textcolor[rgb]{0.68,0.00,0.00}{#1}}
\newcommand{\RegionMarkerTok}[1]{\textcolor[rgb]{0.00,0.23,0.31}{#1}}
\newcommand{\SpecialCharTok}[1]{\textcolor[rgb]{0.37,0.37,0.37}{#1}}
\newcommand{\SpecialStringTok}[1]{\textcolor[rgb]{0.13,0.47,0.30}{#1}}
\newcommand{\StringTok}[1]{\textcolor[rgb]{0.13,0.47,0.30}{#1}}
\newcommand{\VariableTok}[1]{\textcolor[rgb]{0.07,0.07,0.07}{#1}}
\newcommand{\VerbatimStringTok}[1]{\textcolor[rgb]{0.13,0.47,0.30}{#1}}
\newcommand{\WarningTok}[1]{\textcolor[rgb]{0.37,0.37,0.37}{\textit{#1}}}

\providecommand{\tightlist}{%
  \setlength{\itemsep}{0pt}\setlength{\parskip}{0pt}}\usepackage{longtable,booktabs,array}
\usepackage{calc} % for calculating minipage widths
% Correct order of tables after \paragraph or \subparagraph
\usepackage{etoolbox}
\makeatletter
\patchcmd\longtable{\par}{\if@noskipsec\mbox{}\fi\par}{}{}
\makeatother
% Allow footnotes in longtable head/foot
\IfFileExists{footnotehyper.sty}{\usepackage{footnotehyper}}{\usepackage{footnote}}
\makesavenoteenv{longtable}
\usepackage{graphicx}
\makeatletter
\def\maxwidth{\ifdim\Gin@nat@width>\linewidth\linewidth\else\Gin@nat@width\fi}
\def\maxheight{\ifdim\Gin@nat@height>\textheight\textheight\else\Gin@nat@height\fi}
\makeatother
% Scale images if necessary, so that they will not overflow the page
% margins by default, and it is still possible to overwrite the defaults
% using explicit options in \includegraphics[width, height, ...]{}
\setkeys{Gin}{width=\maxwidth,height=\maxheight,keepaspectratio}
% Set default figure placement to htbp
\makeatletter
\def\fps@figure{htbp}
\makeatother

\KOMAoption{captions}{tableheading}
\makeatletter
\makeatother
\makeatletter
\makeatother
\makeatletter
\@ifpackageloaded{caption}{}{\usepackage{caption}}
\AtBeginDocument{%
\ifdefined\contentsname
  \renewcommand*\contentsname{Table of contents}
\else
  \newcommand\contentsname{Table of contents}
\fi
\ifdefined\listfigurename
  \renewcommand*\listfigurename{List of Figures}
\else
  \newcommand\listfigurename{List of Figures}
\fi
\ifdefined\listtablename
  \renewcommand*\listtablename{List of Tables}
\else
  \newcommand\listtablename{List of Tables}
\fi
\ifdefined\figurename
  \renewcommand*\figurename{Figure}
\else
  \newcommand\figurename{Figure}
\fi
\ifdefined\tablename
  \renewcommand*\tablename{Table}
\else
  \newcommand\tablename{Table}
\fi
}
\@ifpackageloaded{float}{}{\usepackage{float}}
\floatstyle{ruled}
\@ifundefined{c@chapter}{\newfloat{codelisting}{h}{lop}}{\newfloat{codelisting}{h}{lop}[chapter]}
\floatname{codelisting}{Listing}
\newcommand*\listoflistings{\listof{codelisting}{List of Listings}}
\makeatother
\makeatletter
\@ifpackageloaded{caption}{}{\usepackage{caption}}
\@ifpackageloaded{subcaption}{}{\usepackage{subcaption}}
\makeatother
\makeatletter
\@ifpackageloaded{tcolorbox}{}{\usepackage[many]{tcolorbox}}
\makeatother
\makeatletter
\@ifundefined{shadecolor}{\definecolor{shadecolor}{rgb}{.97, .97, .97}}
\makeatother
\makeatletter
\makeatother
\ifLuaTeX
  \usepackage{selnolig}  % disable illegal ligatures
\fi
\IfFileExists{bookmark.sty}{\usepackage{bookmark}}{\usepackage{hyperref}}
\IfFileExists{xurl.sty}{\usepackage{xurl}}{} % add URL line breaks if available
\urlstyle{same} % disable monospaced font for URLs
\hypersetup{
  pdftitle={Problem Set 1},
  pdfauthor={Yipeng Liu},
  colorlinks=true,
  linkcolor={blue},
  filecolor={Maroon},
  citecolor={Blue},
  urlcolor={Blue},
  pdfcreator={LaTeX via pandoc}}

\title{Problem Set 1}
\author{Yipeng Liu}
\date{}

\begin{document}
\maketitle
\ifdefined\Shaded\renewenvironment{Shaded}{\begin{tcolorbox}[sharp corners, interior hidden, frame hidden, borderline west={3pt}{0pt}{shadecolor}, enhanced, boxrule=0pt, breakable]}{\end{tcolorbox}}\fi

\hypertarget{homework-1}{%
\section{Homework 1}\label{homework-1}}

\hypertarget{problem-1}{%
\subsection{Problem 1}\label{problem-1}}

\hypertarget{step-1}{%
\subsubsection{Step 1}\label{step-1}}

First, convert the .data file into the data frame form and give
appropriate column names.

\begin{Shaded}
\begin{Highlighting}[]
\CommentTok{\# Import the data from the .data file}
\NormalTok{wineDf }\OtherTok{\textless{}{-}} \FunctionTok{read.table}\NormalTok{(}\StringTok{"/Users/yipengliu/Desktop/Graduate Courses/STATS 506/Homework/Problem Set 1/wine/wine.data"}\NormalTok{, }\AttributeTok{sep =} \StringTok{","}\NormalTok{)}
\FunctionTok{colnames}\NormalTok{(wineDf) }\OtherTok{\textless{}{-}} \FunctionTok{c}\NormalTok{(}\StringTok{"Type"}\NormalTok{, }
                      \StringTok{"Alcohol"}\NormalTok{, }
                      \StringTok{"MalicAcid"}\NormalTok{, }
                      \StringTok{"Ash"}\NormalTok{,}
                      \StringTok{"AlcalinityOfAsh"}\NormalTok{,}
                      \StringTok{"Magnesium"}\NormalTok{,}
                      \StringTok{"TotalPhenols"}\NormalTok{,}
                      \StringTok{"Flavanoids"}\NormalTok{,}
                      \StringTok{"NonflavanoidPhenols"}\NormalTok{,}
                      \StringTok{"Proanthocyanins"}\NormalTok{,}
                      \StringTok{"ColorIntensity"}\NormalTok{,}
                      \StringTok{"Hue"}\NormalTok{,}
                      \StringTok{"OD280OD315OfDilutedWines"}\NormalTok{,}
                      \StringTok{"Proline"}\NormalTok{)}

\CommentTok{\# Display the first 5 rows of the data frame}
\FunctionTok{head}\NormalTok{(wineDf, }\DecValTok{5}\NormalTok{)}
\end{Highlighting}
\end{Shaded}

\begin{verbatim}
  Type Alcohol MalicAcid  Ash AlcalinityOfAsh Magnesium TotalPhenols Flavanoids
1    1   14.23      1.71 2.43            15.6       127         2.80       3.06
2    1   13.20      1.78 2.14            11.2       100         2.65       2.76
3    1   13.16      2.36 2.67            18.6       101         2.80       3.24
4    1   14.37      1.95 2.50            16.8       113         3.85       3.49
5    1   13.24      2.59 2.87            21.0       118         2.80       2.69
  NonflavanoidPhenols Proanthocyanins ColorIntensity  Hue
1                0.28            2.29           5.64 1.04
2                0.26            1.28           4.38 1.05
3                0.30            2.81           5.68 1.03
4                0.24            2.18           7.80 0.86
5                0.39            1.82           4.32 1.04
  OD280OD315OfDilutedWines Proline
1                     3.92    1065
2                     3.40    1050
3                     3.17    1185
4                     3.45    1480
5                     2.93     735
\end{verbatim}

\hypertarget{step-2}{%
\subsubsection{Step 2}\label{step-2}}

Check the wine with the highest and lowest alcohol content belongs.

\begin{Shaded}
\begin{Highlighting}[]
\CommentTok{\# Type of wine contains the highest alcohol content}
\NormalTok{typeContainsMostAlcohol }\OtherTok{\textless{}{-}} 
\NormalTok{  wineDf}\SpecialCharTok{$}\NormalTok{Type[}\FunctionTok{which}\NormalTok{(wineDf}\SpecialCharTok{$}\NormalTok{Alcohol }\SpecialCharTok{==} \FunctionTok{max}\NormalTok{(wineDf}\SpecialCharTok{$}\NormalTok{Alcohol))]}
\NormalTok{typeContainsMostAlcohol}
\end{Highlighting}
\end{Shaded}

\begin{verbatim}
[1] 1
\end{verbatim}

\begin{Shaded}
\begin{Highlighting}[]
\CommentTok{\# Type of wine contains the lowest alcohol content}
\NormalTok{typeContainsLowestAlcohol }\OtherTok{\textless{}{-}} 
\NormalTok{  wineDf}\SpecialCharTok{$}\NormalTok{Type[}\FunctionTok{which}\NormalTok{(wineDf}\SpecialCharTok{$}\NormalTok{Alcohol }\SpecialCharTok{==} \FunctionTok{min}\NormalTok{(wineDf}\SpecialCharTok{$}\NormalTok{Alcohol))]}
\NormalTok{typeContainsMostAlcohol}
\end{Highlighting}
\end{Shaded}

\begin{verbatim}
[1] 1
\end{verbatim}

It follows that the wine with the highest alcohol content belongs to
class 1, and that the wine with the lowest alcohol content also belongs
to class 1.

Then, explore how many wines in our example have higher levels of
magnesium than average German beers.

\begin{Shaded}
\begin{Highlighting}[]
\NormalTok{quantityHigherThanGerman }\OtherTok{\textless{}{-}} \FunctionTok{sum}\NormalTok{(wineDf}\SpecialCharTok{$}\NormalTok{Magnesium }\SpecialCharTok{\textgreater{}} \DecValTok{114}\NormalTok{)}
\NormalTok{quantityHigherThanGerman}
\end{Highlighting}
\end{Shaded}

\begin{verbatim}
[1] 26
\end{verbatim}

It follows that there are 26 wines have higher levels of magnesium than
German beers.

Finally, find out within each class, how many wines have higher levels
of magnesium than average German beers.

\begin{Shaded}
\begin{Highlighting}[]
\CommentTok{\# Create a new column for the categorial variable "whether of not it has higher levels of magnesium than German beers"}
\NormalTok{wineDf}\SpecialCharTok{$}\NormalTok{HigherThanGermanBeers }\OtherTok{\textless{}{-}} 
  \FunctionTok{as.numeric}\NormalTok{(wineDf}\SpecialCharTok{$}\NormalTok{Magnesium }\SpecialCharTok{\textgreater{}} \DecValTok{114}\NormalTok{)}

\CommentTok{\# Calculate within each class, how many wines have higher levels of magnesium than average German beers}
\NormalTok{sumHigherThanGermanBeers }\OtherTok{\textless{}{-}} 
  \FunctionTok{tapply}\NormalTok{(wineDf}\SpecialCharTok{$}\NormalTok{HigherThanGermanBeers, }
         \AttributeTok{INDEX =}\NormalTok{ wineDf}\SpecialCharTok{$}\NormalTok{Type, }
         \AttributeTok{FUN =}\NormalTok{ sum)}
\NormalTok{sumHigherThanGermanBeers}
\end{Highlighting}
\end{Shaded}

\begin{verbatim}
 1  2  3 
15  6  5 
\end{verbatim}

It can be seen that 15, 6, and 5 wines in categories 1, 2, and 3,
respectively, have higher magnesium contents than average German beers.

\hypertarget{step-3}{%
\subsubsection{Step 3}\label{step-3}}

Here creates a table identifying the average value of each variable,
providing one row for the overall average, and one row per class with
class averages.

\begin{Shaded}
\begin{Highlighting}[]
\CommentTok{\# Get the average of each column}
\NormalTok{colMeansWine }\OtherTok{\textless{}{-}} \FunctionTok{colMeans}\NormalTok{(wineDf)}

\CommentTok{\# Get the average of each column per class}
\NormalTok{colMeansWineByClass }\OtherTok{\textless{}{-}} \FunctionTok{aggregate}\NormalTok{(wineDf, }\AttributeTok{by =} \FunctionTok{list}\NormalTok{(wineDf}\SpecialCharTok{$}\NormalTok{Type), }\AttributeTok{FUN =}\NormalTok{ mean)}
\NormalTok{colMeansWineByClass }\OtherTok{\textless{}{-}}\NormalTok{ colMeansWineByClass[, }\SpecialCharTok{{-}}\DecValTok{1}\NormalTok{]}

\CommentTok{\# Bind these results into a single data frame}
\NormalTok{aveWineDf1 }\OtherTok{\textless{}{-}} \FunctionTok{data.frame}\NormalTok{(}\FunctionTok{t}\NormalTok{(colMeansWine))}
\NormalTok{aveWineDf }\OtherTok{\textless{}{-}} \FunctionTok{rbind}\NormalTok{(colMeansWineByClass, aveWineDf1)}

\CommentTok{\# Show the data frame}
\FunctionTok{head}\NormalTok{(aveWineDf)}
\end{Highlighting}
\end{Shaded}

\begin{verbatim}
      Type  Alcohol MalicAcid      Ash AlcalinityOfAsh Magnesium TotalPhenols
1 1.000000 13.74475  2.010678 2.455593        17.03729 106.33898     2.840169
2 2.000000 12.27873  1.932676 2.244789        20.23803  94.54930     2.258873
3 3.000000 13.15375  3.333750 2.437083        21.41667  99.31250     1.678750
4 1.938202 13.00062  2.336348 2.366517        19.49494  99.74157     2.295112
  Flavanoids NonflavanoidPhenols Proanthocyanins ColorIntensity       Hue
1  2.9823729           0.2900000        1.899322       5.528305 1.0620339
2  2.0808451           0.3636620        1.630282       3.086620 1.0562817
3  0.7814583           0.4475000        1.153542       7.396250 0.6827083
4  2.0292697           0.3618539        1.590899       5.058090 0.9574494
  OD280OD315OfDilutedWines   Proline HigherThanGermanBeers
1                 3.157797 1115.7119            0.25423729
2                 2.785352  519.5070            0.08450704
3                 1.683542  629.8958            0.10416667
4                 2.611685  746.8933            0.14606742
\end{verbatim}

In this data frame we can see that rows 1, 2, and 3 represent the
average of each variable for beers of type 1, 2, and 3, respectively,
and the last row represents the average of each variable for all wines.

\hypertarget{step-4}{%
\subsubsection{Step 4}\label{step-4}}

Now carry out a series of t-tests to examine whether the level of Ash
differs across the three classes.

\begin{Shaded}
\begin{Highlighting}[]
\CommentTok{\# First, get 3 lists that represents the level of Ash of the 3 different kinds of wine}
\NormalTok{ash1 }\OtherTok{\textless{}{-}} \FunctionTok{subset}\NormalTok{(wineDf}\SpecialCharTok{$}\NormalTok{Ash, wineDf}\SpecialCharTok{$}\NormalTok{Type }\SpecialCharTok{==} \DecValTok{1}\NormalTok{)}
\NormalTok{ash2 }\OtherTok{\textless{}{-}} \FunctionTok{subset}\NormalTok{(wineDf}\SpecialCharTok{$}\NormalTok{Ash, wineDf}\SpecialCharTok{$}\NormalTok{Type }\SpecialCharTok{==} \DecValTok{2}\NormalTok{)}
\NormalTok{ash3 }\OtherTok{\textless{}{-}} \FunctionTok{subset}\NormalTok{(wineDf}\SpecialCharTok{$}\NormalTok{Ash, wineDf}\SpecialCharTok{$}\NormalTok{Type }\SpecialCharTok{==} \DecValTok{3}\NormalTok{)}

\CommentTok{\# Then, carry out the t{-}tests for 1\textasciitilde{}2, 2\textasciitilde{}3, 1\textasciitilde{}3}
\FunctionTok{t.test}\NormalTok{(ash1, ash2)}
\end{Highlighting}
\end{Shaded}

\begin{verbatim}

    Welch Two Sample t-test

data:  ash1 and ash2
t = 4.4184, df = 125.59, p-value = 2.124e-05
alternative hypothesis: true difference in means is not equal to 0
95 percent confidence interval:
 0.116383 0.305226
sample estimates:
mean of x mean of y 
 2.455593  2.244789 
\end{verbatim}

\begin{Shaded}
\begin{Highlighting}[]
\FunctionTok{t.test}\NormalTok{(ash2, ash3)}
\end{Highlighting}
\end{Shaded}

\begin{verbatim}

    Welch Two Sample t-test

data:  ash2 and ash3
t = -4.184, df = 114.96, p-value = 5.627e-05
alternative hypothesis: true difference in means is not equal to 0
95 percent confidence interval:
 -0.2833328 -0.1012564
sample estimates:
mean of x mean of y 
 2.244789  2.437083 
\end{verbatim}

\begin{Shaded}
\begin{Highlighting}[]
\FunctionTok{t.test}\NormalTok{(ash1, ash3)}
\end{Highlighting}
\end{Shaded}

\begin{verbatim}

    Welch Two Sample t-test

data:  ash1 and ash3
t = 0.46489, df = 105, p-value = 0.643
alternative hypothesis: true difference in means is not equal to 0
95 percent confidence interval:
 -0.06043717  0.09745695
sample estimates:
mean of x mean of y 
 2.455593  2.437083 
\end{verbatim}

In the t-test of these 3 pairs of independent samples, we can observe
that the p-value of the t-test results of the 2 pairs of samples,
1\textasciitilde2 and 2\textasciitilde3, is much less than 0.05, so the
null hypothesis can be rejected and it is considered that there is a
significant difference between each of the 2 pairs of samples.

On the contrary, the t-test for the pair of samples 1~3 showed a
p-value greater than 0.05, so the null hypothesis could not be rejected
and the pair of samples could not be considered significantly different
from each other.

\hypertarget{problem-2}{%
\subsection{Problem 2}\label{problem-2}}

\hypertarget{step-1-1}{%
\subsubsection{Step 1}\label{step-1-1}}

First, write the function.

\begin{Shaded}
\begin{Highlighting}[]
\CommentTok{\#\textquotesingle{} This function identifies whether a given integer is a "perfect power\textquotesingle{}}
\CommentTok{\#\textquotesingle{} }
\CommentTok{\#\textquotesingle{} @param number the number to check}
\CommentTok{\#\textquotesingle{} @param power the specified power}
\CommentTok{\#\textquotesingle{} }
\CommentTok{\#\textquotesingle{} @return the list that shows the result}
\NormalTok{isPerfectPower }\OtherTok{\textless{}{-}} \ControlFlowTok{function}\NormalTok{(number, power) \{}
  
  \ControlFlowTok{if}\NormalTok{ (}\FunctionTok{abs}\NormalTok{(}\FunctionTok{round}\NormalTok{(number)}\SpecialCharTok{{-}}\NormalTok{number }\SpecialCharTok{!=} \DecValTok{0}\NormalTok{) }\SpecialCharTok{\textgreater{}=}\NormalTok{ .Machine}\SpecialCharTok{$}\NormalTok{double.eps}\SpecialCharTok{\^{}}\FloatTok{0.5}\NormalTok{) \{}
    \FunctionTok{stop}\NormalTok{(}\StringTok{"The number must be an integer."}\NormalTok{)}
\NormalTok{  \}}
  \ControlFlowTok{if}\NormalTok{ (}\FunctionTok{abs}\NormalTok{(}\FunctionTok{round}\NormalTok{(power)}\SpecialCharTok{{-}}\NormalTok{power }\SpecialCharTok{!=} \DecValTok{0}\NormalTok{) }\SpecialCharTok{\textgreater{}=}\NormalTok{ .Machine}\SpecialCharTok{$}\NormalTok{double.eps}\SpecialCharTok{\^{}}\FloatTok{0.5}\NormalTok{) \{}
    \FunctionTok{stop}\NormalTok{(}\StringTok{"The power must be an integer."}\NormalTok{)}
\NormalTok{  \}}
  
\NormalTok{  isPerfect }\OtherTok{=} \FunctionTok{abs}\NormalTok{(}\FunctionTok{round}\NormalTok{(number}\SpecialCharTok{\^{}}\NormalTok{(}\DecValTok{1}\SpecialCharTok{/}\NormalTok{power))}\SpecialCharTok{{-}}\NormalTok{number}\SpecialCharTok{\^{}}\NormalTok{(}\DecValTok{1}\SpecialCharTok{/}\NormalTok{power)) }\SpecialCharTok{\textless{}}\NormalTok{ .Machine}\SpecialCharTok{$}\NormalTok{double.eps}\SpecialCharTok{\^{}}\FloatTok{0.5}
\NormalTok{  root }\OtherTok{=}\NormalTok{ number}\SpecialCharTok{\^{}}\NormalTok{(}\DecValTok{1}\SpecialCharTok{/}\NormalTok{power)}
  
  \FunctionTok{return}\NormalTok{(}\FunctionTok{list}\NormalTok{(isPerfect, root))}
\NormalTok{\}}
\end{Highlighting}
\end{Shaded}

\hypertarget{step-2-1}{%
\subsubsection{Step 2}\label{step-2-1}}

Next, demonstrate our function works.

\begin{Shaded}
\begin{Highlighting}[]
\CommentTok{\#\textquotesingle{} This function which calls the function above, using a loop to identify both the root and power}
\CommentTok{\#\textquotesingle{} }
\CommentTok{\#\textquotesingle{} @param number the input number}
\CommentTok{\#\textquotesingle{} }
\CommentTok{\#\textquotesingle{} @return the root and the power as a valid string equation with an appropriate message if the input is not a perfect power}
\NormalTok{findRootPower }\OtherTok{\textless{}{-}} \ControlFlowTok{function}\NormalTok{(number) \{}
  
\NormalTok{  myVar }\OtherTok{\textless{}{-}} \ConstantTok{FALSE}
\NormalTok{  powers }\OtherTok{\textless{}{-}} \FunctionTok{c}\NormalTok{()}
  
  \ControlFlowTok{for}\NormalTok{ (i }\ControlFlowTok{in} \DecValTok{2}\SpecialCharTok{:}\FunctionTok{round}\NormalTok{(}\FunctionTok{sqrt}\NormalTok{(number))) \{}
    \ControlFlowTok{if}\NormalTok{ (}\FunctionTok{isPerfectPower}\NormalTok{(number, i)[[}\DecValTok{1}\NormalTok{]] }\SpecialCharTok{==} \ConstantTok{TRUE}\NormalTok{) \{}
\NormalTok{      myVar }\OtherTok{\textless{}{-}} \ConstantTok{TRUE}
\NormalTok{      powers }\OtherTok{\textless{}{-}} \FunctionTok{append}\NormalTok{(powers, i)}
\NormalTok{    \}}
\NormalTok{  \}}
  
  \ControlFlowTok{if}\NormalTok{ (myVar }\SpecialCharTok{==} \ConstantTok{TRUE}\NormalTok{) \{}
\NormalTok{    selectedPower }\OtherTok{=} \FunctionTok{min}\NormalTok{(powers)}
\NormalTok{    myString }\OtherTok{=} \FunctionTok{paste}\NormalTok{(}\FunctionTok{as.character}\NormalTok{(number), }
                     \StringTok{" = "}\NormalTok{,}
                     \FunctionTok{as.character}\NormalTok{(number}\SpecialCharTok{\^{}}\NormalTok{(}\DecValTok{1}\SpecialCharTok{/}\NormalTok{selectedPower)),}
                     \StringTok{"\^{}"}\NormalTok{, }
                     \FunctionTok{as.character}\NormalTok{(selectedPower))}
    \FunctionTok{return}\NormalTok{(myString)}
\NormalTok{  \}}
  \ControlFlowTok{else}\NormalTok{ \{}
    \FunctionTok{return}\NormalTok{(}\StringTok{"The number entered is not a perfect power."}\NormalTok{)}
\NormalTok{  \}}
\NormalTok{\}}
\end{Highlighting}
\end{Shaded}

\hypertarget{step-3-1}{%
\subsubsection{Step 3}\label{step-3-1}}

Finally, use this function to test on some examples.

\begin{Shaded}
\begin{Highlighting}[]
\FunctionTok{findRootPower}\NormalTok{(}\DecValTok{27}\NormalTok{)}
\end{Highlighting}
\end{Shaded}

\begin{verbatim}
[1] "27  =  3 ^ 3"
\end{verbatim}

\begin{Shaded}
\begin{Highlighting}[]
\FunctionTok{findRootPower}\NormalTok{(}\DecValTok{13060694016}\NormalTok{)}
\end{Highlighting}
\end{Shaded}

\begin{verbatim}
[1] "13060694016  =  6 ^ 13"
\end{verbatim}

\begin{Shaded}
\begin{Highlighting}[]
\FunctionTok{findRootPower}\NormalTok{(}\DecValTok{7776}\NormalTok{)}
\end{Highlighting}
\end{Shaded}

\begin{verbatim}
[1] "7776  =  6 ^ 5"
\end{verbatim}

\begin{Shaded}
\begin{Highlighting}[]
\FunctionTok{findRootPower}\NormalTok{(}\DecValTok{170859375}\NormalTok{)}
\end{Highlighting}
\end{Shaded}

\begin{verbatim}
[1] "170859375  =  15 ^ 7"
\end{verbatim}

\begin{Shaded}
\begin{Highlighting}[]
\FunctionTok{findRootPower}\NormalTok{(}\DecValTok{58247422}\NormalTok{)}
\end{Highlighting}
\end{Shaded}

\begin{verbatim}
[1] "The number entered is not a perfect power."
\end{verbatim}

\begin{Shaded}
\begin{Highlighting}[]
\FunctionTok{findRootPower}\NormalTok{(}\DecValTok{94143178827}\NormalTok{)}
\end{Highlighting}
\end{Shaded}

\begin{verbatim}
[1] "94143178827  =  3 ^ 23"
\end{verbatim}

\hypertarget{problem-3}{%
\subsection{Problem 3}\label{problem-3}}

\hypertarget{step-1-2}{%
\subsubsection{Step 1}\label{step-1-2}}

Here is the answer from GPT-4:

\emph{Here are two R functions that accomplish the tasks you described:}

\begin{Shaded}
\begin{Highlighting}[]
\CommentTok{\# Function to determine the name of a poker hand}
\NormalTok{determine\_poker\_hand }\OtherTok{\textless{}{-}} \ControlFlowTok{function}\NormalTok{(suits, ranks) \{}
  \CommentTok{\# Count the occurrences of each rank}
\NormalTok{  rank\_counts }\OtherTok{\textless{}{-}} \FunctionTok{table}\NormalTok{(ranks)}
  
  \CommentTok{\# Check for a flush (all cards have the same suit)}
\NormalTok{  is\_flush }\OtherTok{\textless{}{-}} \FunctionTok{length}\NormalTok{(}\FunctionTok{unique}\NormalTok{(suits)) }\SpecialCharTok{==} \DecValTok{1}
  
  \CommentTok{\# Check for a straight (consecutive ranks)}
\NormalTok{  is\_straight }\OtherTok{\textless{}{-}} \FunctionTok{max}\NormalTok{(ranks) }\SpecialCharTok{{-}} \FunctionTok{min}\NormalTok{(ranks) }\SpecialCharTok{==} \DecValTok{4} \SpecialCharTok{\&\&} \FunctionTok{length}\NormalTok{(}\FunctionTok{unique}\NormalTok{(ranks)) }\SpecialCharTok{==} \DecValTok{5}
  
  \CommentTok{\# Determine the hand name}
  \ControlFlowTok{if}\NormalTok{ (is\_flush }\SpecialCharTok{\&\&}\NormalTok{ is\_straight) \{}
    \FunctionTok{return}\NormalTok{(}\StringTok{"Straight Flush"}\NormalTok{)  }\CommentTok{\# Both straight and flush}
\NormalTok{  \} }\ControlFlowTok{else} \ControlFlowTok{if}\NormalTok{ (}\FunctionTok{any}\NormalTok{(rank\_counts }\SpecialCharTok{==} \DecValTok{4}\NormalTok{)) \{}
    \FunctionTok{return}\NormalTok{(}\StringTok{"Four of a Kind"}\NormalTok{)  }\CommentTok{\# Have 4 cards with the same kind}
\NormalTok{  \} }\ControlFlowTok{else} \ControlFlowTok{if}\NormalTok{ (}\FunctionTok{all}\NormalTok{(rank\_counts }\SpecialCharTok{==} \FunctionTok{c}\NormalTok{(}\DecValTok{2}\NormalTok{, }\DecValTok{3}\NormalTok{))) \{}
    \FunctionTok{return}\NormalTok{(}\StringTok{"Full House"}\NormalTok{)  }\CommentTok{\# Have 3 cards with identical rank and 2 cards with another identical rank}
\NormalTok{  \} }\ControlFlowTok{else} \ControlFlowTok{if}\NormalTok{ (is\_flush) \{}
    \FunctionTok{return}\NormalTok{(}\StringTok{"Flush"}\NormalTok{)}
\NormalTok{  \} }\ControlFlowTok{else} \ControlFlowTok{if}\NormalTok{ (is\_straight) \{}
    \FunctionTok{return}\NormalTok{(}\StringTok{"Straight"}\NormalTok{)  }
\NormalTok{  \} }\ControlFlowTok{else} \ControlFlowTok{if}\NormalTok{ (}\FunctionTok{any}\NormalTok{(rank\_counts }\SpecialCharTok{==} \DecValTok{3}\NormalTok{)) \{}
    \FunctionTok{return}\NormalTok{(}\StringTok{"Three of a Kind"}\NormalTok{)  }\CommentTok{\# Have 3 cards with identical kind}
\NormalTok{  \} }\ControlFlowTok{else} \ControlFlowTok{if}\NormalTok{ (}\FunctionTok{sum}\NormalTok{(rank\_counts }\SpecialCharTok{==} \DecValTok{2}\NormalTok{) }\SpecialCharTok{==} \DecValTok{4}\NormalTok{) \{}
    \FunctionTok{return}\NormalTok{(}\StringTok{"Two Pair"}\NormalTok{)  }\CommentTok{\# Have 2 cards with identical rank and 2 cards with another identical rank}
\NormalTok{  \} }\ControlFlowTok{else} \ControlFlowTok{if}\NormalTok{ (}\FunctionTok{sum}\NormalTok{(rank\_counts }\SpecialCharTok{==} \DecValTok{2}\NormalTok{) }\SpecialCharTok{==} \DecValTok{2}\NormalTok{) \{}
    \FunctionTok{return}\NormalTok{(}\StringTok{"One Pair"}\NormalTok{)  }\CommentTok{\# Have only 1 pair of cards with identical rank}
\NormalTok{  \} }\ControlFlowTok{else}\NormalTok{ \{}
    \FunctionTok{return}\NormalTok{(}\StringTok{"High Card"}\NormalTok{)}
\NormalTok{  \}}
\NormalTok{\}}

\CommentTok{\# Function to simulate dealing a round of poker}
\NormalTok{deal\_poker\_round }\OtherTok{\textless{}{-}} \ControlFlowTok{function}\NormalTok{(num\_players) \{}
  \CommentTok{\# Define the deck of cards}
\NormalTok{  ranks }\OtherTok{\textless{}{-}} \FunctionTok{rep}\NormalTok{(}\DecValTok{2}\SpecialCharTok{:}\DecValTok{14}\NormalTok{, }\DecValTok{4}\NormalTok{) }\CommentTok{\# 2 through Ace (14), each repeated 4 times}
\NormalTok{  suits }\OtherTok{\textless{}{-}} \FunctionTok{rep}\NormalTok{(}\FunctionTok{c}\NormalTok{(}\StringTok{"Hearts"}\NormalTok{, }\StringTok{"Diamonds"}\NormalTok{, }\StringTok{"Clubs"}\NormalTok{, }\StringTok{"Spades"}\NormalTok{), }\AttributeTok{each =} \DecValTok{13}\NormalTok{)}
\NormalTok{  deck }\OtherTok{\textless{}{-}} \FunctionTok{data.frame}\NormalTok{(}\AttributeTok{Rank =}\NormalTok{ ranks, }\AttributeTok{Suit =}\NormalTok{ suits)  }\CommentTok{\# Combine the suits and ranks within a single data frame}
  \CommentTok{\# Shuffle the deck}
  \FunctionTok{set.seed}\NormalTok{(}\DecValTok{506}\NormalTok{)  }\CommentTok{\# This line is added by myself to make the results deterministic}
\NormalTok{  shuffled\_deck }\OtherTok{\textless{}{-}}\NormalTok{ deck[}\FunctionTok{sample}\NormalTok{(}\FunctionTok{nrow}\NormalTok{(deck)), ]}
  
  \CommentTok{\# Deal cards to each player}
\NormalTok{  player\_hands }\OtherTok{\textless{}{-}} \FunctionTok{list}\NormalTok{()  }\CommentTok{\# Initialize a empty list to store the cards of each player}
  \ControlFlowTok{for}\NormalTok{ (i }\ControlFlowTok{in} \DecValTok{1}\SpecialCharTok{:}\NormalTok{num\_players) \{}
\NormalTok{    player\_hand }\OtherTok{\textless{}{-}}\NormalTok{ shuffled\_deck[((i }\SpecialCharTok{{-}} \DecValTok{1}\NormalTok{) }\SpecialCharTok{*} \DecValTok{5} \SpecialCharTok{+} \DecValTok{1}\NormalTok{)}\SpecialCharTok{:}\NormalTok{(i }\SpecialCharTok{*} \DecValTok{5}\NormalTok{), ]  }\CommentTok{\# The numbers are taken sequentially from the data frame, and the process is treated as dealing cards to each players}
\NormalTok{    player\_hands[[i]] }\OtherTok{\textless{}{-}}\NormalTok{ player\_hand  }\CommentTok{\# append each player\textquotesingle{}s cards to the list}
    \FunctionTok{cat}\NormalTok{(}\StringTok{"Player"}\NormalTok{, i, }\StringTok{"hand:"}\NormalTok{, }\StringTok{"}\SpecialCharTok{\textbackslash{}n}\StringTok{"}\NormalTok{)  }\CommentTok{\# Output the player\textquotesingle{}s cards}
    \FunctionTok{print}\NormalTok{(player\_hand)}
\NormalTok{    hand\_name }\OtherTok{\textless{}{-}} \FunctionTok{determine\_poker\_hand}\NormalTok{(player\_hand}\SpecialCharTok{$}\NormalTok{Suit, player\_hand}\SpecialCharTok{$}\NormalTok{Rank)}
    \FunctionTok{cat}\NormalTok{(}\StringTok{"Hand Name:"}\NormalTok{, hand\_name, }\StringTok{"}\SpecialCharTok{\textbackslash{}n\textbackslash{}n}\StringTok{"}\NormalTok{)  }\CommentTok{\# Output the player\textquotesingle{}s name of the kind}
\NormalTok{  \}}
\NormalTok{\}}

\CommentTok{\# Example usage:}
\FunctionTok{deal\_poker\_round}\NormalTok{(}\DecValTok{4}\NormalTok{)  }\CommentTok{\# Simulate a round of poker with 4 players}
\end{Highlighting}
\end{Shaded}

\begin{verbatim}
Player 1 hand: 
   Rank   Suit
48   10 Spades
3     4 Hearts
1     2 Hearts
43    5 Spades
27    2  Clubs
Hand Name: High Card 

Player 2 hand: 
   Rank     Suit
22   10 Diamonds
5     6   Hearts
10   11   Hearts
20    8 Diamonds
8     9   Hearts
\end{verbatim}

\begin{verbatim}
Warning in rank_counts == c(2, 3): longer object length is not a multiple of
shorter object length
\end{verbatim}

\begin{verbatim}
Hand Name: High Card 

Player 3 hand: 
   Rank   Suit
51   13 Spades
36   11  Clubs
32    7  Clubs
34    9  Clubs
49   11 Spades
Hand Name: High Card 

Player 4 hand: 
   Rank   Suit
39   14  Clubs
11   12 Hearts
40    2 Spades
37   12  Clubs
29    4  Clubs
Hand Name: High Card 
\end{verbatim}

\emph{These functions allow you to determine the name of a poker hand
based on a given set of suits and ranks, as well as simulate dealing a
round of poker with a specified number of players, displaying each
player's hand and its corresponding name.}

\hypertarget{step-2-2}{%
\subsubsection{Step 2}\label{step-2-2}}

Test the code with a range of inputs.

\begin{Shaded}
\begin{Highlighting}[]
\FunctionTok{deal\_poker\_round}\NormalTok{(}\DecValTok{2}\NormalTok{)}
\end{Highlighting}
\end{Shaded}

\begin{verbatim}
Player 1 hand: 
   Rank   Suit
48   10 Spades
3     4 Hearts
1     2 Hearts
43    5 Spades
27    2  Clubs
Hand Name: High Card 

Player 2 hand: 
   Rank     Suit
22   10 Diamonds
5     6   Hearts
10   11   Hearts
20    8 Diamonds
8     9   Hearts
\end{verbatim}

\begin{verbatim}
Warning in rank_counts == c(2, 3): longer object length is not a multiple of
shorter object length
\end{verbatim}

\begin{verbatim}
Hand Name: High Card 
\end{verbatim}

\begin{Shaded}
\begin{Highlighting}[]
\FunctionTok{deal\_poker\_round}\NormalTok{(}\DecValTok{3}\NormalTok{)}
\end{Highlighting}
\end{Shaded}

\begin{verbatim}
Player 1 hand: 
   Rank   Suit
48   10 Spades
3     4 Hearts
1     2 Hearts
43    5 Spades
27    2  Clubs
Hand Name: High Card 

Player 2 hand: 
   Rank     Suit
22   10 Diamonds
5     6   Hearts
10   11   Hearts
20    8 Diamonds
8     9   Hearts
\end{verbatim}

\begin{verbatim}
Warning in rank_counts == c(2, 3): longer object length is not a multiple of
shorter object length
\end{verbatim}

\begin{verbatim}
Hand Name: High Card 

Player 3 hand: 
   Rank   Suit
51   13 Spades
36   11  Clubs
32    7  Clubs
34    9  Clubs
49   11 Spades
Hand Name: High Card 
\end{verbatim}

\begin{Shaded}
\begin{Highlighting}[]
\FunctionTok{deal\_poker\_round}\NormalTok{(}\DecValTok{5}\NormalTok{)}
\end{Highlighting}
\end{Shaded}

\begin{verbatim}
Player 1 hand: 
   Rank   Suit
48   10 Spades
3     4 Hearts
1     2 Hearts
43    5 Spades
27    2  Clubs
Hand Name: High Card 

Player 2 hand: 
   Rank     Suit
22   10 Diamonds
5     6   Hearts
10   11   Hearts
20    8 Diamonds
8     9   Hearts
\end{verbatim}

\begin{verbatim}
Warning in rank_counts == c(2, 3): longer object length is not a multiple of
shorter object length
\end{verbatim}

\begin{verbatim}
Hand Name: High Card 

Player 3 hand: 
   Rank   Suit
51   13 Spades
36   11  Clubs
32    7  Clubs
34    9  Clubs
49   11 Spades
Hand Name: High Card 

Player 4 hand: 
   Rank   Suit
39   14  Clubs
11   12 Hearts
40    2 Spades
37   12  Clubs
29    4  Clubs
Hand Name: High Card 

Player 5 hand: 
   Rank     Suit
24   12 Diamonds
25   13 Diamonds
6     7   Hearts
2     3   Hearts
44    6   Spades
\end{verbatim}

\begin{verbatim}
Warning in rank_counts == c(2, 3): longer object length is not a multiple of
shorter object length
\end{verbatim}

\begin{verbatim}
Hand Name: High Card 
\end{verbatim}

\begin{Shaded}
\begin{Highlighting}[]
\FunctionTok{deal\_poker\_round}\NormalTok{(}\DecValTok{6}\NormalTok{)}
\end{Highlighting}
\end{Shaded}

\begin{verbatim}
Player 1 hand: 
   Rank   Suit
48   10 Spades
3     4 Hearts
1     2 Hearts
43    5 Spades
27    2  Clubs
Hand Name: High Card 

Player 2 hand: 
   Rank     Suit
22   10 Diamonds
5     6   Hearts
10   11   Hearts
20    8 Diamonds
8     9   Hearts
\end{verbatim}

\begin{verbatim}
Warning in rank_counts == c(2, 3): longer object length is not a multiple of
shorter object length
\end{verbatim}

\begin{verbatim}
Hand Name: High Card 

Player 3 hand: 
   Rank   Suit
51   13 Spades
36   11  Clubs
32    7  Clubs
34    9  Clubs
49   11 Spades
Hand Name: High Card 

Player 4 hand: 
   Rank   Suit
39   14  Clubs
11   12 Hearts
40    2 Spades
37   12  Clubs
29    4  Clubs
Hand Name: High Card 

Player 5 hand: 
   Rank     Suit
24   12 Diamonds
25   13 Diamonds
6     7   Hearts
2     3   Hearts
44    6   Spades
\end{verbatim}

\begin{verbatim}
Warning in rank_counts == c(2, 3): longer object length is not a multiple of
shorter object length
\end{verbatim}

\begin{verbatim}
Hand Name: High Card 

Player 6 hand: 
   Rank     Suit
23   11 Diamonds
21    9 Diamonds
15    3 Diamonds
14    2 Diamonds
52   14   Spades
\end{verbatim}

\begin{verbatim}
Warning in rank_counts == c(2, 3): longer object length is not a multiple of
shorter object length
\end{verbatim}

\begin{verbatim}
Hand Name: High Card 
\end{verbatim}

\begin{Shaded}
\begin{Highlighting}[]
\FunctionTok{deal\_poker\_round}\NormalTok{(}\DecValTok{7}\NormalTok{)}
\end{Highlighting}
\end{Shaded}

\begin{verbatim}
Player 1 hand: 
   Rank   Suit
48   10 Spades
3     4 Hearts
1     2 Hearts
43    5 Spades
27    2  Clubs
Hand Name: High Card 

Player 2 hand: 
   Rank     Suit
22   10 Diamonds
5     6   Hearts
10   11   Hearts
20    8 Diamonds
8     9   Hearts
\end{verbatim}

\begin{verbatim}
Warning in rank_counts == c(2, 3): longer object length is not a multiple of
shorter object length
\end{verbatim}

\begin{verbatim}
Hand Name: High Card 

Player 3 hand: 
   Rank   Suit
51   13 Spades
36   11  Clubs
32    7  Clubs
34    9  Clubs
49   11 Spades
Hand Name: High Card 

Player 4 hand: 
   Rank   Suit
39   14  Clubs
11   12 Hearts
40    2 Spades
37   12  Clubs
29    4  Clubs
Hand Name: High Card 

Player 5 hand: 
   Rank     Suit
24   12 Diamonds
25   13 Diamonds
6     7   Hearts
2     3   Hearts
44    6   Spades
\end{verbatim}

\begin{verbatim}
Warning in rank_counts == c(2, 3): longer object length is not a multiple of
shorter object length
\end{verbatim}

\begin{verbatim}
Hand Name: High Card 

Player 6 hand: 
   Rank     Suit
23   11 Diamonds
21    9 Diamonds
15    3 Diamonds
14    2 Diamonds
52   14   Spades
\end{verbatim}

\begin{verbatim}
Warning in rank_counts == c(2, 3): longer object length is not a multiple of
shorter object length
\end{verbatim}

\begin{verbatim}
Hand Name: High Card 

Player 7 hand: 
   Rank     Suit
35   10    Clubs
18    6 Diamonds
26   14 Diamonds
50   12   Spades
19    7 Diamonds
\end{verbatim}

\begin{verbatim}
Warning in rank_counts == c(2, 3): longer object length is not a multiple of
shorter object length
\end{verbatim}

\begin{verbatim}
Hand Name: High Card 
\end{verbatim}

\hypertarget{step-3-2}{%
\subsubsection{Step 3}\label{step-3-2}}

I have already explained the code line by line in Step 1.

\hypertarget{step-4-1}{%
\subsubsection{Step 4}\label{step-4-1}}

The function ``determine\_poker\_hand'' that ChatGPT generated above
have 2 problems: a. it allows more than 10 hands to be dealt; b. it will
generate the warning: ``Warning: longer object length is not a multiple
of shorter object length''.

Fore example:

\begin{Shaded}
\begin{Highlighting}[]
\FunctionTok{deal\_poker\_round}\NormalTok{(}\DecValTok{12}\NormalTok{)}
\end{Highlighting}
\end{Shaded}

\begin{verbatim}
Player 1 hand: 
   Rank   Suit
48   10 Spades
3     4 Hearts
1     2 Hearts
43    5 Spades
27    2  Clubs
Hand Name: High Card 

Player 2 hand: 
   Rank     Suit
22   10 Diamonds
5     6   Hearts
10   11   Hearts
20    8 Diamonds
8     9   Hearts
\end{verbatim}

\begin{verbatim}
Warning in rank_counts == c(2, 3): longer object length is not a multiple of
shorter object length
\end{verbatim}

\begin{verbatim}
Hand Name: High Card 

Player 3 hand: 
   Rank   Suit
51   13 Spades
36   11  Clubs
32    7  Clubs
34    9  Clubs
49   11 Spades
Hand Name: High Card 

Player 4 hand: 
   Rank   Suit
39   14  Clubs
11   12 Hearts
40    2 Spades
37   12  Clubs
29    4  Clubs
Hand Name: High Card 

Player 5 hand: 
   Rank     Suit
24   12 Diamonds
25   13 Diamonds
6     7   Hearts
2     3   Hearts
44    6   Spades
\end{verbatim}

\begin{verbatim}
Warning in rank_counts == c(2, 3): longer object length is not a multiple of
shorter object length
\end{verbatim}

\begin{verbatim}
Hand Name: High Card 

Player 6 hand: 
   Rank     Suit
23   11 Diamonds
21    9 Diamonds
15    3 Diamonds
14    2 Diamonds
52   14   Spades
\end{verbatim}

\begin{verbatim}
Warning in rank_counts == c(2, 3): longer object length is not a multiple of
shorter object length
\end{verbatim}

\begin{verbatim}
Hand Name: High Card 

Player 7 hand: 
   Rank     Suit
35   10    Clubs
18    6 Diamonds
26   14 Diamonds
50   12   Spades
19    7 Diamonds
\end{verbatim}

\begin{verbatim}
Warning in rank_counts == c(2, 3): longer object length is not a multiple of
shorter object length
\end{verbatim}

\begin{verbatim}
Hand Name: High Card 

Player 8 hand: 
   Rank     Suit
38   13    Clubs
16    4 Diamonds
13   14   Hearts
12   13   Hearts
4     5   Hearts
Hand Name: High Card 

Player 9 hand: 
   Rank     Suit
42    4   Spades
45    7   Spades
47    9   Spades
17    5 Diamonds
41    3   Spades
\end{verbatim}

\begin{verbatim}
Warning in rank_counts == c(2, 3): longer object length is not a multiple of
shorter object length
\end{verbatim}

\begin{verbatim}
Hand Name: High Card 

Player 10 hand: 
   Rank   Suit
31    6  Clubs
33    8  Clubs
7     8 Hearts
46    8 Spades
30    5  Clubs
\end{verbatim}

\begin{verbatim}
Warning in rank_counts == c(2, 3): longer object length is not a multiple of
shorter object length
\end{verbatim}

\begin{verbatim}
Hand Name: Three of a Kind 

Player 11 hand: 
     Rank   Suit
9      10 Hearts
28      3  Clubs
NA     NA   <NA>
NA.1   NA   <NA>
NA.2   NA   <NA>
Hand Name: High Card 

Player 12 hand: 
     Rank Suit
NA     NA <NA>
NA.1   NA <NA>
NA.2   NA <NA>
NA.3   NA <NA>
NA.4   NA <NA>
Hand Name: Full House 
\end{verbatim}

It shows that the 11th player is only assigned 2 cards, and that the
12th player is not assigned any cards.

To revise the function and fix the problems, I made following attempts:

\begin{Shaded}
\begin{Highlighting}[]
\NormalTok{deal\_poker\_round\_revised }\OtherTok{\textless{}{-}} \ControlFlowTok{function}\NormalTok{(num\_players) \{}
  \ControlFlowTok{if}\NormalTok{ (num\_players }\SpecialCharTok{\textless{}=} \DecValTok{10}\NormalTok{) \{}
\NormalTok{    ranks }\OtherTok{\textless{}{-}} \FunctionTok{rep}\NormalTok{(}\DecValTok{2}\SpecialCharTok{:}\DecValTok{14}\NormalTok{, }\DecValTok{4}\NormalTok{) }
\NormalTok{    suits }\OtherTok{\textless{}{-}} \FunctionTok{rep}\NormalTok{(}\FunctionTok{c}\NormalTok{(}\StringTok{"Hearts"}\NormalTok{, }\StringTok{"Diamonds"}\NormalTok{, }\StringTok{"Clubs"}\NormalTok{, }\StringTok{"Spades"}\NormalTok{), }\AttributeTok{each =} \DecValTok{13}\NormalTok{)}
\NormalTok{    deck }\OtherTok{\textless{}{-}} \FunctionTok{data.frame}\NormalTok{(}\AttributeTok{Rank =}\NormalTok{ ranks, }\AttributeTok{Suit =}\NormalTok{ suits)  }
  
    \FunctionTok{set.seed}\NormalTok{(}\DecValTok{506}\NormalTok{)  }
\NormalTok{    shuffled\_deck }\OtherTok{\textless{}{-}}\NormalTok{ deck[}\FunctionTok{sample}\NormalTok{(}\FunctionTok{nrow}\NormalTok{(deck)), ]}
  
\NormalTok{    player\_hands }\OtherTok{\textless{}{-}} \FunctionTok{list}\NormalTok{() }
  
    \ControlFlowTok{for}\NormalTok{ (i }\ControlFlowTok{in} \DecValTok{1}\SpecialCharTok{:}\NormalTok{num\_players) \{}
\NormalTok{      player\_hand }\OtherTok{\textless{}{-}}\NormalTok{ shuffled\_deck[((i }\SpecialCharTok{{-}} \DecValTok{1}\NormalTok{) }\SpecialCharTok{*} \DecValTok{5} \SpecialCharTok{+} \DecValTok{1}\NormalTok{)}\SpecialCharTok{:}\NormalTok{(i }\SpecialCharTok{*} \DecValTok{5}\NormalTok{), ]  }
\NormalTok{      player\_hands[[i]] }\OtherTok{\textless{}{-}}\NormalTok{ player\_hand }
      \FunctionTok{cat}\NormalTok{(}\StringTok{"Player"}\NormalTok{, i, }\StringTok{"hand:"}\NormalTok{, }\StringTok{"}\SpecialCharTok{\textbackslash{}n}\StringTok{"}\NormalTok{) }
      \FunctionTok{print}\NormalTok{(player\_hand)}
\NormalTok{      hand\_name }\OtherTok{\textless{}{-}} \FunctionTok{determine\_poker\_hand}\NormalTok{(player\_hand}\SpecialCharTok{$}\NormalTok{Suit, player\_hand}\SpecialCharTok{$}\NormalTok{Rank)}
      \FunctionTok{cat}\NormalTok{(}\StringTok{"Hand Name:"}\NormalTok{, hand\_name, }\StringTok{"}\SpecialCharTok{\textbackslash{}n\textbackslash{}n}\StringTok{"}\NormalTok{) }
\NormalTok{    \}}
\NormalTok{  \}}
  \ControlFlowTok{else}\NormalTok{ \{}
    \FunctionTok{print}\NormalTok{(}\StringTok{"The number of players exceeds the limit."}\NormalTok{)}
\NormalTok{  \}}
\NormalTok{\}}
\end{Highlighting}
\end{Shaded}

Then we can check what will it happen if the number of players exceeds
the limit.

\begin{Shaded}
\begin{Highlighting}[]
\FunctionTok{deal\_poker\_round\_revised}\NormalTok{(}\DecValTok{11}\NormalTok{)}
\end{Highlighting}
\end{Shaded}

\begin{verbatim}
[1] "The number of players exceeds the limit."
\end{verbatim}

As for the warning ``Warning: longer object length is not a multiple of
shorter object length'', I'm still working on it.



\end{document}
